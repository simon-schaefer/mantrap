\section{Statement of Contributions}
\label{text:introduction/contributions}
This thesis presents a trajectory optimization formulation for reliable, safe and un-disturbing navigation among pedestrians, while exploiting multi-modal and probabilistic behavior predictions and intrinsically leveraging prediction-model internal interactive knowledge.
\newline\newline
The contributions of the thesis are:

\begin{itemize}
\item Demonstrating the potentials of tightly-integrating deep data-driven, generative models in trajectory optimization, at the example of the Trajectron \cite{Ivanovic2018}\cite{Salzmann2020}. Therefore, we firstly evince the feasibility of couple such a model for online optimization, secondly expound an algorithm design that leverages the models internal knowledge representation and thirdly, we formulate an objective function design that exploits the probabilistic, multi-modal and long-horizon output of the generative model effectively and without further simplification by leveraging prior work in loss functions for training data-driven models.  
\item Using this techniques, we present a trajectory optimization formulation for socially-aware navigation among pedestrians based on multi-modal and probabilistic pedestrian predictions; focussing on the disturbance the robot's trajectory will introduce on the pedestrians as well as provable safety. In particular, a new way of pedestrian-centric trajectory optimization is introduced, explicitly targeting a socially un-disturbing way of interaction between robot and humans. For augmenting the range pedestrian prediction models, that can be combined with the presented formulation, an algorithm for transforming deterministic and short-horizon models into generative trajectory prediction models is provided.
\item Results in free-space planar simulation demonstrate the approach and benchmark it against other state-of-the-art solutions tackling socially-aware navigation.
\end{itemize}