\chapter{Related Work}
\label{text:related_work}


\begin{figure}[!ht]
\begin{center}
\includegraphics[width=\imgwidth]{images/placeholder.png}
\captionof{figure}{Example predictions of Trajectron model}
\label{img:trajectron_example}
\end{center}
\end{figure}

%Although there are many different approaches for tackling different parts of the problem, such as predicting and simulating human movement, finding a collision-free trajectory or doing both in an end-to-end manner using deep reinforcement learning techniques, the problem is still far from being solved and applicable to a wide range of real-world scenarios.


% todo: goals of the work
% todo: do not list papers !!!  a lot of different methods for planning in crowds different axes: interpretable optimisation, accurate behaviour prediction, but find general topics e.g. 
% 1. prediction of pedestrians (Trajectron, SF, ORCA, ...) 
% 2. control \& decision making (interaction-aware papers for example manipulation but multiple agents,  explainability, certifiability, ...)
% 3. types of trajectory optimisation (shooting, collocation, etc.)
% 4. safety (safety in interaction e.g. using reachability $\rightarrow$ Mathias althoff); what is good and bad about them ? 
% example: related work: Karen \& Boris journal paper, Eds papers for motivation
% citations: Haruki's work (limitations), decision making for interactive scenario (Dragan)