\chapter{Conclusion and Future Work}
\label{text:conclusion}

\section{Conclusion}
\label{text:conclusions/conclusions}
In this thesis, I designed and implemented a new approach of socially-aware trajectory optimization for crowd-navigation. In this field, purely data-driven methods lack tractability, while model-driven systems usually miss the predictive power of learned generative models. The purposed optimization formulation combines the strengths of both methodologies and is seamlessly integrating into probabilistic, multi-model, state-of-the-art prediction models. Thereby, it harnesses its predicted distributions over future pedestrian states precisely and without the need for sampling methods.
\newline\newline
By directly leveraging the prediction model's internal structure, the interactive objective function drives the optimization to (socially) anticipative but cautious solutions. An \ac{HJR}-based constraint further guarantees interaction safety within the planning horizon, without being overly conservative.
\newline\newline
Several strategies have been substantiated and implemented to efficiently solve the designed optimization using the interior-point method-based solver \ac{IPOPT}. However, the goal of running at $10 Hz$ could not have been reached.
\newline\newline
Finally, I presented results and validated the final algorithm for an assorted set of pedestrian prediction models in simulation. A sampled-based algorithm has been proposed to compute an output distribution for deterministic, parametric models. In comparison to other state-of-the-art, both data-driven as well as conventional trajectory planning algorithms, the purposed approach shows great performance.
\newline

\section{Future Work}
\label{text:conclusions/future_work}
In the following, I list possible extensions and future research directions.

\begin{itemize}
\item For fully integrating the purposed planning methodology into real-world applications, several external determinants have to be considered in planning. While undoubtedly the most challenging factor, the interaction with dynamic agents, has been treated within this work, integrating the structure of the environment itself (such as sidewalks, roads, static-obstacles), remains for future work. Since most prediction models such as Trajectron \cite{Salzmann2020} do already support including the environment structure, doing so probably is straight-forward. As previously shown, most data-driven prediction models have been trained on data recorded in structured environments, leading to (unintentional) artifacts in their prediction. Adding structure to the environment, therefore, may even improve the performance of the overall approach.
\item The bottleneck of the purposed approach is the number of evaluations of the interactive objective and its gradient. \ac{SQP} has shown to require less objective function evaluations compared to interior-point methods, at the cost of intermediate infeasibility. To further improve the algorithm's efficiency, it might be reasonable to use a \ac{SQP}-based solver such as Gusto \cite{Bonalli2019} instead.
\item Due to the lack of robot-conditioned prediction models, the algorithm could only be validated for a selected set of models. A future research direction may condition other state-of-the-art pedestrian prediction models on the planned robot's trajectory, such as \ac{SGAN} \cite{Gupta2018}, and integrate these into the planning framework. This includes evaluating the framework on other "socially-aware" agent types, such as bicyclists or cars. Therefore, in theory, no non-parametric change in the overall formulation is required.
\item During this work, it is assumed for the robot to underly double integrator dynamics. A generalization of the method to other robot dynamics should be straight-forward.
\item Finally, the algorithm has been tested in simulation only, mainly due to the implications of COVID-19. While several efforts have been introduced to mimic the challenges of a real-world environment as best as possible, they obviously could not perfectly be matched. Thus, despite the discussed drawbacks of real-world testing in terms of comparability of the results, it remains for future work to test it in a real-world environment. 
\end{itemize}

The interactive optimization design fully leveraging the predictive power of a learned deep neural network has never been done before for trajectory optimization, to the best of my knowledge. With the rise of data-driven prediction models, it gains increasing importance to unleash their full predictive power in optimization in many fields. Therefore, this work can be seen as a proof-of-concept of applying this concept in trajectory optimization and many other optimization-driven fields, such as adaptive control or manufacturing optimization. In contrast to purely learning-based approaches using online optimization gives rise to the possibility of constraint the solution in a tractable manner. Due to the ongoing trend of neural network compression in many fields, computational bounds of the purposed approach will fade. With an increasing amount of computational complex simulation being replaced with data-driven approaches, this work may be one of the foundations for applying neural network gradients in online optimization.