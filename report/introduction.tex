\chapter{Introduction and Motivation}
\label{text:introduction}
Autonomous systems play an increasingly fundamental role in our society. A study of McKinsey in 2017 showed the large effects increasing automatic will have on the economy; one of the keys inside of the study is that even as automation causes declines in some occupations, it will change many more, with estimated 60 percent of occupations of which at least 30 percent of constituent work activities will be automated \footnote{\href{https://www.mckinsey.com/~/media/McKinsey/Featured\%20Insights/Future\%20of\%20Organizations/What\%20the\%20future\%20of\%20work\%20will\%20mean\%20for\%20jobs\%20skills\%20and\%20wages/MGI-Jobs-Lost-Jobs-Gained-Report-December-6-2017.ashx}{Jobs lost, Jobs gained: Workforce transitions in a time of automation - McKinsey Global International}}. The International Federation of Robotics, an association of unilateral robotics societies and companies operating in the field, illustrated the rise of service robots in many fields, such as logistics, healthcare or agriculture, in its annual report 2019.\footnote{\href{https://ifr.org/downloads/press2018/IFR\%20World\%20Robotics\%20Presentation\%20-\%2018\%20Sept\%202019.pdf}{Annual Report 2019 - IFR}} Next to the field of fully autonomous robots there also they state an increasing share of collaborative robots, e.g. assisting object retrieval in warehouses or cleaning in hospitals.

\begin{figure}[!ht]
\begin{center}
\includegraphics[width=\imgwidth]{images/service_robotics.png}
\captionof{figure}{Estimated global revenue of service robots by professional use from 2017 until 2020}
\label{img:service_robotics}
\end{center}
\end{figure}

A blog article published by the Management Review of MIT's Sloan School stated that this development might be vastly accelerated due to the societal effects of COVID-19.\footnote{\href{https://sloanreview.mit.edu/article/ai-robots-and-ethics-in-the-age-of-covid-19/}{AI, Robots, and Ethics in the Age of COVID-19 - MIT Sloan Management Review}} A perfect example of this development can be observed in the Bishan-Ang Mo Kio Park in Singapore, where the robot dog "Spot" is patrolling to monitor social distancing measures. While in the example the robot still is piloted by a human controller, in future these systems should work fully autonomously.\footnote{\href{http://www.straitstimes.com/singapore/meet-spot-the-safe-distancing-robot-on-trial-in-bishan-amk-park}{Meet Spot, the safe distancing robot on trial in Bishan-AMK Park - The Straits Times}}

\begin{figure}[!ht]
\begin{center}
\includegraphics[width=\imgwidth]{images/spot.jpg}
\captionof{figure}{Boston Dynamics robot dog Spot patrolling in Bishan-Ang Mo Kio Park in Singapore in May 2020}
\label{img:spot_in_park}
\end{center}
\end{figure}

Summarising, robots taking part in a growing part of our social and economical life, however, efficient but more importantly safe interaction between robots and humans is crucial for its success, interactions such as walking in between them as in the example displaying in Figure \ref{img:spot_in_park}.
\newline
In interactive multi-agent scenarios, the standard control loop of perceiving the sensor input, predicting, planning, and executing the action \cite{Siegwart2011} is extended by a backward loop, due to the effects of the planned actions which have to be taken into account. Therefore after planning some future trajectory of the robot, a simulation predicts the behavior of the interacting agents, conditioned on the planned robot trajectory (comp. Figure \ref{fig:control_loop_interactive}, the idea from \cite{Romanski2019}). To guarantee interaction properties such safety usually the planner has to take into account the behavior of the other agents. However, as the term interactive implies, this behavior depends on the planned robot trajectory. This intrinsic dependence loop makes it especially hard to deal with the interactive scenario. 

\begin{figure}
\begin{center}
\tikzstyle{block} = [draw, rectangle, minimum height=3em, minimum width=6em]
\tikzstyle{sim} = [draw, fill=cyan, rectangle, minimum height=3em, minimum width=6em]
\tikzstyle{input} = [draw, fill=orange, rectangle, minimum height=3em, minimum width=6em]
\tikzstyle{output} = [draw, fill=yellow, rectangle, minimum height=3em, minimum width=6em]
\begin{tikzpicture}[align=center, node distance=3cm,>=latex']

    \node[input, name=input](input){Sensor Input};
    \node [block, right of=input] (perception){Perception};
    \draw [->] (input) -- node {} (perception);
    \node [block, right of=perception] (prediction){Prediction};
    \draw [->] (perception) -- node[name=x] {} (prediction);
    \node [block, right of=prediction] (planning){Planning};
    \draw [->] (prediction) -- node[name=y] {} (planning);
    \node [sim, below of=y] (simulation){Simulation};
    \node [output, right of=planning] (actors){Actors};
    \draw [->] (planning) -- node[name=z] {} (actors);
    \draw [->] (z) |- (simulation);
    \draw [->] (simulation) -| (x);

\end{tikzpicture}
\captionof{figure}{Robotics control loop for interactive multi-agent scenarios}
\label{fig:control_loop_interactive}
\end{center}
\end{figure}

Within the field of safe navigation in crowded areas, especially in for human crowds. While humans use their "theory of mind", for reasoning about the other human's actions \cite{Gweon2013}, robots do not have this capacity (so far). As a consequence approaches that act in this highly dynamic, uncertain, and multi-modal environment try to reduce the complexity of the problem by making strong assumptions about human behavior and/or by avoiding any risk. These approaches lack at least one of these properties, that are crucial for working reliable and safe near humans: 

\begin{itemize}
\item Realistic representation of the real, complex human behavior including its multi-modal, probabilistic and highly dynamic nature
\item Certifiability by being able to explain and predict the outcome (or at least bounding it) to guarantee the safety 
\item Computational efficient for real-time application
\end{itemize}

When walking in crowded areas humans use their theory of mind, giving them the capability to infer the actions of oncoming agents \cite{Ivanovic2018} \cite{Gweon2013}. To navigate through the crowd while avoiding the others as best as possible then using the inferred information the human steers into local minima regarding both the required travel time to reach some goal and avoiding the other pedestrian. In this way, as everyday life shows, the human way of walking naturally combines efficiency and safety. Although safety measures are not explicitly "optimized" collisions are prevented by reducing the amount of interaction, so that possibly un-safe scenarios are omitted before they can evolve instead of trying to contain safety within the interaction. Thus, in this project, this behavior should be imitated by minimizing both the estimated travel time and a cost for minimizing the interaction in parallel, while being informed by a "theory-of-mind"-like the pedestrian prediction model. 


\section{Goals}
\label{text:introduction/goals}
In this work, I want to tackle safe navigation in human crowds by combining two of the main paradigms in this field, trajectory optimization algorithms, and deep learning prediction models. I want to come up with an algorithm that both takes the complex human behavior into account, without the need of strong assumptions about human behavior, while still being able to track and guarantee properties of the resulting robot trajectory. Also, I want to re-frame the problem of safe navigation from pure travel-time or safety optimally to an interaction-aware objective, minimizing the disturbance the robot causes on the human behavior, e.g. to not disrupt doctors in a hospital floor or the pedestrians strolling in the park mentioned above, while guaranteeing safety and reaching the goal when possible.
\newline
Therefore I will use a state-of-the-art deep learned probabilistic and multi-modal pedestrian trajectory prediction model combined with a shooting trajectory optimization algorithm. To inform the gradients of its objective function, I will not use the model's output but the prediction models internal structure itself to optimally exploit the "hidden" information about the interactions with other pedestrians and the robot that are stored in the latent representations of the deeply learned model. Thereby the main objective of the algorithm is not to reach the goal in the smallest possible robot travel-time, but to disturb the surrounding humans as less as feasible while still reaching the robot's goal. Additionally, a Hamilton-Jacobi reachability based constraint will be used for guaranteeing a safe interaction. The algorithm should be real-time feasible running with a frequency of $10 Hz$, similar to related work discussed in chapter \ref{text:related_work} (e.g. \cite{Chen2017}).


\section{Outline of Work}
\label{text:introduction/outline}
This thesis contains five parts. Chapter \ref{text:introduction} motivates this work and provides insights into the challenges as well as possible applications, defines its goals, and gives a brief overview of the solution developed within the thesis. Chapter \ref{text:related_work} introduces the reader into related work in the areas of the pedestrian prediction model, controls \& decision-making algorithms focussing on the interaction between robots and humans, types of trajectory optimization algorithms and to technical notions of safety. Chapter \ref{text:approach} deep dives into the algorithm developed within this work, starting with a more detailed summary and continuing with a detailed description and explanation of each of its parts. Chapter \ref{text:experiments} validates the approach by presenting experimental results for different settings and scenarios, measuring its performance on baselines, and discussing the results. Finally, chapter \ref{text:conclusion} concludes the work and gives an outlook for future research directions.

\section{Statement of Contributions}
\label{text:introduction/contributions}

%todo: statement of contributions
