\section{Evaluation Metrics}
\label{text:experiments/metrics}
While several related works such as \cite{Everett2018}\cite{Vemula2016}\cite{Luo2018a}\cite{Phillips2011} primarily use the average number of collisions over multiple runs (the "success rate") to measure the performance of their navigation algorithm, other works such as \cite{Chen2017}\cite{Nishimura2020a} considers more manifold metrics. Thus, there is no common way of comparing the capabilities of an algorithm in the field of socially-aware navigation. As the performance measures used in \cite{Chen2017},  which are similar to the ones used in \cite{Nishimura2020a}, seem to be the most descriptive out of the reviewed related work, they will be the base for the conducted experiments. In \cite{Chen2017} Chen et al.\ present three performance metrics: 

\begin{enumerate}
\item average extra time to reach the goals $t_{extra}$, which compares the time each agent needs to get from its initial position $\boldsymbol{p}^i_0$ to some imaginary (or pre-defined) goal position $\boldsymbol{p}^i_g$ in absence of any other agent with the time it takes in presence of other agents, running the underlying algorithm.

\begin{equation}
t_{extra} = \frac{1}{N_{agents}} \sum_{i=0}^{N_{agents}} \left( t_g^i - \frac{||\boldsymbol{p}^i_0 - \boldsymbol{p}^i_g||_2}{v^k_{pref}} \right)
\label{eq:t_extra}
\end{equation}

\item minimum separation distance, which simply points out the minimal distance between two agents at any time.
\item relative preference between left-handedness and right-handedness, which indicates the amount of violation in \cite{Chen2017}'s customly defined social laws of desired behavior.
\end{enumerate}

Due to their generality beyond the approach used by Chen et alt.\, the first and the second can be adapted to be of use here. While the first metric, the average extra time, focusses on the efficiency of the derived solution, the second metric examines safety aspects. However in order to disentangle the orientation and the control effort of the robot, the first metric is divided into two separate metrics. \\

\begin{figure}[!ht]
\begin{center}
\begin{tikzpicture}

    \node[] (ped) at (1, 1.5)
    {\includegraphics[width=.05\textwidth]{images/walking.png}};
    \draw[velocity] (ped) to (0.5, -1);
    
    \node[] (robot) at (-4,0)
    {\includegraphics[width=.05\textwidth]{images/robot.png}};
    \node[] (goal) at (5,0)
    {\includegraphics[width=.05\textwidth]{images/flag.png}};
       
    \draw [loosely dotted, ultra thick] (robot) to[out=0, in=180] node[below]{$A$} (1, 0);
    \draw [densely dotted, ultra thick] (1, 0) to (goal);
    \draw [densely dotted, ultra thick] (robot) to[out=0, in=180] node[above, sloped]{$B$} (0, 2.5) to[out=0] (goal);
    
\end{tikzpicture}
\end{center}
\caption{Comparison of two trajectories $A$ and $B$ to reach the goal while avoiding a pedestrian. Following trajectory $A$ the robot starts moving slowly in the direction of the goal until the pedestrian has passed, and is accelerating afterwards. Following trajectory $B$ the robot has a constant large speed, but points to the goal not before it has passed the pedestrian.}
\label{img:extra_time}
\end{figure}

While the trajectories $A$ and $B$ in Figure \ref{img:extra_time} might introduce the same value extra travel time$t_{extra}(A) = t_{extra}(B)$, they trade-off a constant high speed ($B$) with a goal-directed but partly low-speed ($A$) robot behavior. So, although depicting very different approaches, their assigned value  $t_{extra}$ might be equal. Therefore, within project \project the "directness" of the resulting robot trajectory is regarded separately from the control effort the robot requires to reach its goal. Additionally, as we explicitly target to minimize the impact the robot's motion has on the pedestrian's behavior, the set of evaluation metrics is expanded to entail the additional effort a pedestrian is expected to incur.  

\subsection{Robot Control Effort (RCE)}
The RCE metric is a classical objective in control theory, measuring the control energy that the robot uses to perform the resulting trajectory. Although not optimized directly, the RCE is an important metric for assessing the quality and feasibility of the solution trajectory in real-world applications. For normalization, the required control energy is compared to the maximal accessible control input $u_{max}$ over the length of the solution trajectory $\stepssolution$:

\begin{equation}
RCE = \frac{\sum_{i = 0}^{\stepssolution} ||\u_i||_2}{\stepssolution \cdot u_{max}}
\end{equation}

\subsection{Robot Trajectory Directness (RTD)}
As previously discussed the RTD metric is introduced to un-couple evaluating the robot's orientation and speed in the solution trajectory. Thereby, the RTD determines the amount of the robot's velocity that target into the goal direction, averaged over the length of the solution trajectory $\stepssolution$. With $\boldsymbol{s}_i^{goal}$ the unit direction vector pointing from the robot's position at time-step $i$ to the goal position, and $\dx^i$ its velocity vector at this point of time, we get:

\begin{equation}
RTD = \frac{\sum_{i=0}^{\stepssolution} \boldsymbol{s}_i^{goal} \cdot \dx_i}{\stepssolution}
\end{equation}

\subsection{Minimal Separation Distance (MSD)}
The MSD metric accounts for the minimal distance between robot and any pedestrian at any time and therefore manifests the "safeness" of the solution trajectory, e.g.\ at the presence of model uncertainty. Since the underlying simulation is discretized with an integration time-step, that is feasible for online optimization and prediction, the inter-time-step behavior is finely linearly interpolated, to account for the time in between time-steps.

\begin{equation}
MSD = \min_{\substack{\forall k \in K, \\ \forall i \in \stepssolution_{interp}}}(||\x_i - \xped[k]_i||_2) 
\end{equation}

\subsection{Extra Travel Time (ETT)}
The ETT indicates the additional travel time the robot requires to reach its goal, compared to going straight from initial to target state. The time-optimal solution $\stepssolution_{direct} \cdot \dt$ is evaluated by solving a similar simplified optimization problem as Problem \ref{problem:general_warm_starting}, i.e.\ just taking into account the robot's dynamics constraints as well as a goal-driven objective function. Compared to Equation \ref{eq:t_extra} ETT by using a simplified trajectory optimization to retrieve the optimal time solution, instead of assuming some preferred velocity $v_{pref}$, the robot's dynamics are taken into account. On this way the metric is not distorted by delays are non-interaction based effects, such as an initial acceleration period.  

\begin{equation}
ETT = (\stepssolution - \stepssolution_{direct})\dt
\end{equation} 

\subsection{Mean Pedestrian Effort (MPE)}
While most above metrics are explicitly related to the demands the derived solution trajectory poses on the robot, a main goal of the underlying formulation is to reduce the impact the robot's movement has on the "walking comfort" of the surrounding pedestrians. As reasoned in Section \ref{text:approach/objective/interactive} the comfort can be regarded as a function of their accelerational changes, which might occur due to the presence of the robot. The MPE takes account for this "comfort" by comparing the mean acceleration conditioned on the solution trajectory with the mean acceleration without the presence of the robot at all, averaged over the length of the solution trajectory $\stepssolution$ and the number of pedestrian in the scene:

\begin{equation}
MPE = \frac{1}{\stepssolution} \frac{1}{K} \sum_{i=0}^{\stepssolution} \sum_{k=0}^K ||\ddxped[k]_i - \ddxpedwo[k]_i||_2
\end{equation}

\subsection{Terminal Goal Distance (TGD)}
The solver has to balance multiple objectives, mainly the minimizing the distance to its goal state while minimally disturbing the surrounding pedestrian's motions. While the second part of the objective has been captured by the MPE and MSD metric, the first part is assessed by the TGD metric. TGD simply compares the initial goal distance with the final goal distance of a robot trajectory, to be independent of the absolute distance and therefore comparable over different scenarios:

\begin{equation}
TGD = \frac{||\x_T - \goal||_2}{||\x_0 - \goal||_2}
\end{equation}

\subsection{Mean Solving Iterations (MSI)}
The MSI is a metric for comparing the effectiveness of two formulations of an optimization problem, which are being solved using the same solver and the same initial scenario. Basically, MSI is the mean number of solving iterations over multiple experiments.

\subsection{Mean 95 \% Objective Decay (M95OD)}
Similarly to the MSI metric, the M95OD evaluates the convergence speed of an optimization problem, for the same solver and initial scenario. While MSI determines the average number of iterations to solve an optimization up to the (locally) optimal solution, M95OD is the mean of solver iterations until the objective value has decreased by 95 \% of the range between initial and terminal value.
